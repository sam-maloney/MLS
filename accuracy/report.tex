\documentclass{article}
\usepackage[margin=1in]{geometry}
\usepackage{graphicx}
\usepackage{mathtools}
\usepackage{amssymb}
\usepackage{physics}
\usepackage[backend=bibtex, style=nature]{biblatex}
\usepackage[hidelinks]{hyperref}

\newcommand*{\inner}[2]{\ip{\rule[0pt]{0pt}{10pt}#1\,}{\,#2}}
\newcommand*{\mat}[1]{\mathbf{#1}}
\newcommand*{\interp}[1]{\mathcal{I}_h #1}

\addbibresource{C:/Users/samal/Documents/School/Warwick/References/Warwick.bib}

%\includegraphics[width=\textwidth, trim={0 0 0 6.5in},clip]

\begin{document}

\title{
%\vspace{-1in}
%\noindent\makebox[\textwidth]{\includegraphics[width=\paperwidth]{./figures/keyline.pdf}}
FEM and MLS Accuracy}
\author{Sam Maloney}

\maketitle

%\tableofcontents

\noindent
Let angle brackets denote the $L_2$ inner product defined by
\begin{equation}
\inner{u}{v}=\int_\Omega uv\dd{\Omega}.
\end{equation}
The $L_2$ norm then follows as
\begin{equation}
\norm{u}=\norm{u}_{L_2}=\inner{u}{u}^\frac{1}{2}=\left(\int_\Omega u^2\dd{\Omega}\right)^\frac{1}{2}.
\end{equation}
Let $\interp{u}$ be the interpolant of $u$ from the finite element space $V_h$. We consider a subinterval $[x_i, x_{i+1}]$ for $i=0,...,N-1$ where $h_i=x_{i+1}-x_i$ and define a residual function
\begin{equation}
\zeta_i(x)=
\begin{cases}
u(x)-\interp{u}, & x\in[x_i, x_{i+1}]\\
0, & \mathrm{otherwise}
\end{cases}.
\end{equation}
Since $\zeta_i(x_i)=\zeta_i(x_{i+1})=0$ we can write it as the sum of a convergent Fourier sine series on the subinterval
\begin{equation}
\zeta_i(x)=\sum_{k=1}^\infty a_{ki}\sin\left[\frac{k\pi\left(x-x_i\right)}{h_i}\right],
\end{equation}
and therefore
\begin{equation}
\int_{x_i}^{x_{i+1}}[\zeta_i(x)]^2\dd{x}=\frac{h_i}{2}\sum_{k=1}^\infty\abs{a_{ki}}^2.
\end{equation}
Repeated differentiation of $\zeta_i(x)$ then yields
\begin{align}
\int_{x_i}^{x_{i+1}}[\zeta_i'(x)]^2\dd{x}&=\frac{h_i}{2}\sum_{k=1}^\infty\left(\frac{k\pi}{h_i}\right)^2\abs{a_{ki}}^2,\\
\int_{x_i}^{x_{i+1}}[\zeta_i''(x)]^2\dd{x}&=\frac{h_i}{2}\sum_{k=1}^\infty\left(\frac{k\pi}{h_i}\right)^4\abs{a_{ki}}^2,\\
\int_{x_i}^{x_{i+1}}[\zeta_i'''(x)]^2\dd{x}&=\frac{h_i}{2}\sum_{k=1}^\infty\left(\frac{k\pi}{h_i}\right)^6\abs{a_{ki}}^2.
\end{align}
Since $k^6\geq k^4\geq k^2\geq1$ it it clear that
\begin{equation}
\int_{x_i}^{x_{i+1}}[\zeta_i(x)]^2\dd{x}\leq\left(\frac{h_i}{\pi}\right)^2\int_{x_i}^{x_{i+1}}[\zeta_i'(x)]^2\dd{x}\leq\left(\frac{h_i}{\pi}\right)^4\int_{x_i}^{x_{i+1}}[\zeta_i''(x)]^2\dd{x}\leq\left(\frac{h_i}{\pi}\right)^6\int_{x_i}^{x_{i+1}}[\zeta_i'''(x)]^2\dd{x}.
\label{eqn:ordering}
\end{equation}
We then define $h=\max_i h_i$ and sum the residual over the full domain to give
\begin{equation}
\zeta(x)=\sum_{i=0}^{N-1}\zeta_i(x).
\end{equation}


\section{Linear Interpolant}

\noindent
For a linear interpolant $\interp{u}$ one has $(\interp{u})''=0$ everywhere, thus giving $\zeta''=u''-(\interp{u})''=u''$. Therefore, making the substitution $\zeta=u-\interp{u}$ in Eqn.~\ref{eqn:ordering} we have
\begin{equation}
\int_{\Omega}[u-\interp{u}]^2\dd{x}\leq\left(\frac{h}{\pi}\right)^2\int_{\Omega}[u'-(\interp{u})']^2\dd{x}\leq\left(\frac{h}{\pi}\right)^4\int_{\Omega}[u'']^2\dd{x},
\end{equation}
and taking the sqaure root then gives
\begin{equation}
\norm{u-\interp{u}}\leq\frac{h}{\pi}\norm{u'-(\interp{u})'}\leq\left(\frac{h}{\pi}\right)^2\norm{u''}.
\end{equation}


\section{Quadratic Interpolant}

\noindent
For a quadractic interpolant $\interp{u}$ one has $(\interp{u})'''=0$ everywhere, thus giving $\zeta'''=u'''-(\interp{u})'''=u'''$
\begin{equation}
\int_{\Omega}[u-\interp{u}]^2\dd{x}\leq\left(\frac{h}{\pi}\right)^2\int_{\Omega}[u'-(\interp{u})']^2\dd{x}\leq\left(\frac{h}{\pi}\right)^4\int_{\Omega}[u''-(\interp{u})'']^2\dd{x}\leq\left(\frac{h}{\pi}\right)^6\int_{\Omega}[u''']^2\dd{x},
\end{equation}
and taking the sqaure root then gives
\begin{equation}
\norm{u-\interp{u}}\leq\frac{h}{\pi}\norm{u'-(\interp{u})'}\leq\left(\frac{h}{\pi}\right)^2\norm{u''-(\interp{u})''}\leq\left(\frac{h}{\pi}\right)^3\norm{u'''}.
\end{equation}

%\printbibliography[heading=bibintoc]

\end{document}